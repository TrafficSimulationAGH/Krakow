% dokumentacja projektu w czasie tworzenia
\documentclass[a4paper,12pt]{article}
\usepackage[polish]{babel}
\usepackage[utf8]{inputenc}
\usepackage[T1]{fontenc}
\usepackage{indentfirst}
\usepackage[top=2.5cm, bottom=2.5cm, left=2.5cm, right=2.5cm]{geometry}
\usepackage{amsmath}
\usepackage{hyperref}
\usepackage{graphicx}

\title{Symulacja ruchu drogowego na IV obwodnicy Krakowa}
\author{Szymon Gałuszka, Michał Worsowicz, Maciej Nalepa}
\date{\today}
\begin{document}
    \maketitle

    \part{Omówienie projektu}

	\section{Definicja problemu}
	
	Nasz cel to symulacja ruchu drogowego na IV obwodnicy Krakowa \ref{obw}. 
	Konieczna jest definicja tras oraz sposobu poruszania się po nich.
	Na obwodnicy nie ma sygnalizacji świetlnej, skrzyżowania znajdują się najczęściej pod wiaduktem i najpierw należy zjechać z drogi szybkiego ruchu.
	Trasa ma zróżnicowane ograniczenia prędkości oraz różną ilość pasów ruchu.
	
	Ruch drogowy powinien uwzględniać pojazdy osobowe, ciężarowe i transport publiczny.
	Ciężkie pojazdy obowiązują inne ograniczenia prędkości oraz zajmują więcej przestrzeni na jezdni.
	Napływ ruchu powinien odbywać się przez wjazdy na obwodnicę, które wpuszczają samochody z określoną częstotliwością,która może zależeć od pory dnia.
	
	Koniecznym elementem jest symulacja zmiany pasa ruchu. 
	Symulacja może obejmować wydarzenia losowe, takie jak:
	
	\begin{itemize}
		\item blokada pasa ruchu,
		\item zamknięcie zjazdu,
		\item nagłe hamowanie.
    \end{itemize}
    
    \section{Obszar symulacji}
    % Obwodnica IV - szczegóły trasy

    \section{Algorytm}
    % Nagel-Schreckenberg

    \part{Metodologia}

    \section{Struktury danych}
    % Tworzenie symulacji

    \section{Implementacja}
    % Wytłumaczenie rozwiązania problemu

    \section{Usprawnienie}
    % Słabe punkty implementacji, co działa za wolno lub niestabilnie

    \part{Podsumowanie}

    \section{Wyniki}
    % Opis symulacji i wpływu parametrów na model

    \section{Od autorów}
    % Nasze przemyślenia o projekcie, napotkane trudności

    \section{Dalsza praca}
    % Uproszczenia i części wymagające ulepszeń lub optymalizacji
    % Nie umieszczamy ich w projekcie, ale zaznaczamy od czego zacząć aby projekt usprawnić

	\pagebreak
	\begin{thebibliography}{15}
		\bibitem{wikikrk}
		Wiki, \textit{Kraków Obwodnica IV},
		\texttt{\href{https://pl.wikipedia.org/wiki/Obwodnice_Krakowa\#IV_obwodnica}{wikipedia.org}}
		
		\bibitem{map}
		OpenStreet, \textit{mapa},
		\texttt{\href{https://www.openstreetmap.org/}{openstreetmap.org}}
		
		\bibitem{nagel}
		K. Nagel, M. Schreckenberg, \textit{Two lane traffic simulations using cellular automata},
		\texttt{\href{https://arxiv.org/pdf/cond-mat/9512119.pdf}{arxiv.org}}
		% https://www.sciencedirect.com/science/article/pii/0378437195004424
	\end{thebibliography}
	
\end{document}

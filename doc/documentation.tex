% dokumentacja projektu w czasie tworzenia
\documentclass[a4paper,12pt]{article}
\usepackage[polish]{babel}
\usepackage[utf8]{inputenc}
\usepackage[T1]{fontenc}
\usepackage{indentfirst}
\usepackage[top=2.5cm, bottom=2.5cm, left=2.5cm, right=2.5cm]{geometry}
\usepackage{amsmath}
\usepackage{hyperref}
\usepackage{graphicx}

\title{Symulacja ruchu drogowego na IV obwodnicy Krakowa}
\author{Szymon Gałuszka, Michał Worsowicz, Maciej Nalepa}
\date{\today}
\begin{document}
    \maketitle

    \part{Omówienie projektu}

	\section{Definicja problemu}
	
	Nasz cel to symulacja ruchu drogowego na IV obwodnicy Krakowa \ref{obw}. 
	Konieczna jest definicja tras oraz sposobu poruszania się po nich.
	Na obwodnicy nie ma sygnalizacji świetlnej, skrzyżowania znajdują się najczęściej pod wiaduktem i najpierw należy zjechać z drogi szybkiego ruchu.
	Trasa ma zróżnicowane ograniczenia prędkości oraz różną ilość pasów ruchu.
	
	Ruch drogowy powinien uwzględniać pojazdy osobowe, ciężarowe i transport publiczny.
	Ciężkie pojazdy obowiązują inne ograniczenia prędkości oraz zajmują więcej przestrzeni na jezdni.
	Napływ ruchu powinien odbywać się przez wjazdy na obwodnicę, które wpuszczają samochody z określoną częstotliwością, która może zależeć od pory dnia.
	
	Koniecznym elementem jest symulacja zmiany pasa ruchu. 
	Symulacja może obejmować wydarzenia losowe, takie jak:
	
	\begin{itemize}
		\item blokada pasa ruchu,
		\item zamknięcie zjazdu,
		\item nagłe hamowanie.
    \end{itemize}
    
    \section{Obszar symulacji}
    % Obwodnica IV - szczegóły trasy
	Symulowany przez nas obszar to IV obwodnica Krakowa, znana także jako obwodnica autostradowa Krakowa, ponieważ większość jej odcinka stanowi autostrada A4. Fragment ten na zachodnich obrzeżach miasta jest dwupasmowy, a na południowych - trójpasmowy. W najbliższym czasie przewidywana jest budowa odcinka północnego, który postanowiliśmy także uwzględnić w naszej symulacji, odtwarzając i dodając ten fragment do projektu. Poniżej przedstawiony jest wykaz węzłów drogowych, które obejmuje symulacja (zgodnie z kierunkiem ruchu wskazówek zegara):
	
	\begin{itemize}
		\item węzeł Kraków Mistrzejowice (planowany)
		\item węzeł Kraków Grębałów (planowany)
		\item węzeł Kraków Nowa Huta
		\item węzeł Kraków Przewóz
		\item węzeł Kraków Bieżanów
		\item węzeł Kraków Wielicka
		\item węzeł Kraków Łagiewniki
		\item węzeł Kraków Południe
		\item węzeł Kraków Skawina
		\item węzeł Kraków Tyniec
		\item węzeł Kraków Bielany
		\item węzeł Kraków Balice (lotnisko)
		\item węzeł Balice I
		\item węzeł Modlniczka (Rząska)
		\item węzeł Modlnica
		\item węzeł Kraków Zielonki (planowany)
		\item węzeł Kraków Węgrzce (planowany)
		\item węzeł Kraków Batowice (planowany)
	\end{itemize}

    \section{Algorytm}
    % Nagel-Schreckenberg
    Zastosowany przez nas algorytm ruchu drogowego to model Nagela-Schreckenberga, będący teoretycznym modelem mikroskopowym o charakterze dyskretnym, w którym obie jednokierunkowe jezdnie są podzielone na komórki, które mogą z kolei składać się z np. dwóch pasów. Każda z komórek może być interpretowana na jeden z dwóch sposobów:
    
    \begin{itemize}
    	\item fragment pustej drogi
    	\item fragment zawierający pojedyńczy samochód
    \end{itemize}


	Każdy samochód $n$ posiada swoją prędkość $v(n)$ o wartości liczby naturalnej, nie większej od prędkości maksymalnej $v_{max}$ (symbolizuje ona ograniczenie prędkości na drodze).
	Czas $t$ w modelu Nagela-Schreckenberga przyjmuje wartość dyskretną o stałym kroku, gdzie każdy etap można podzielić na cztery następujące po sobie czynności:

	\begin{enumerate}
		\item Przyspieszanie: \\
		Jeżeli $v(n) < v_{max}$ to prędkość samochodu może ulec zwiększeniu o zadaną jednostkę, nie przekraczając prędkości maksymalnej.
		\item Zwalnianie: \\
		Jeżeli odległość $d(n)$ samochodu $n$ od samochodu znajdującego się przed nim jest mniejsza od prędkości $v(n)$ tego samochodu to prędkość samochodu ulega zmniejszeniu. Jako że odległość mierzona jest w liczbie komórek, a prędkość w liczbie komórek na jednostkę czasu, to prędkość ostateczna może wynosić maksymalnie $d(n)$ na jednostkę czasu, a minimalnie zero.
		\item Losowość: \\
		Czynność ta symbolizuje wszelkie przypadki losowe z jakimi kierowca może spotkać się na drodze. Dla każdego samochodu $n$, gdzie $v(n) > 0$, prędkość zostaje zmniejszona o jedną jednostkę, z pewnym zadanym prawdopodobieństwem $p$.
		\item Ruch samochodu: \\
		W ostatnim kroku każdy z samochodów zostaje przesunięty do przodu o odpowiednią ilość komórek, wynikającą z jego prędkości.
	\end{enumerate}

    \part{Metodologia}

    \section{Struktury danych}
    % Tworzenie symulacji

    \section{Implementacja}
    % Wytłumaczenie rozwiązania problemu

    \section{Usprawnienie}
    % Słabe punkty implementacji, co działa za wolno lub niestabilnie

    \part{Podsumowanie}

    \section{Wyniki}
    % Opis symulacji i wpływu parametrów na model
		Wszystko jest gites i działa
    \section{Od autorów}
    % Nasze przemyślenia o projekcie, napotkane trudności
    Pozdrawiamy

    \section{Dalsza praca}
    % Uproszczenia i części wymagające ulepszeń lub optymalizacji
    % Nie umieszczamy ich w projekcie, ale zaznaczamy od czego zacząć aby projekt usprawnić

	\pagebreak
	\begin{thebibliography}{15}
		\bibitem{wikikrk}
		Wiki, \textit{Kraków Obwodnica IV},
		\texttt{\href{https://pl.wikipedia.org/wiki/Obwodnice_Krakowa\#IV_obwodnica}{wikipedia.org}}
		
		\bibitem{map}
		OpenStreet, \textit{mapa},
		\texttt{\href{https://www.openstreetmap.org/}{openstreetmap.org}}
		
		\bibitem{nagel}
		K. Nagel, M. Schreckenberg, \textit{Two lane traffic simulations using cellular automata},
		\texttt{\href{https://arxiv.org/pdf/cond-mat/9512119.pdf}{arxiv.org}}
		% https://www.sciencedirect.com/science/article/pii/0378437195004424
	\end{thebibliography}
	
\end{document}

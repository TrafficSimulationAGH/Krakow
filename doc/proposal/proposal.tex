% Przegląd literatury i ogólnie pojęty research
\documentclass[a4paper,12pt]{article}
\usepackage[polish]{babel}
\usepackage[utf8]{inputenc}
\usepackage[T1]{fontenc}
\usepackage{indentfirst}
\usepackage[top=2.5cm, bottom=2.5cm, left=2.5cm, right=2.5cm]{geometry}
\usepackage{amsmath}
\usepackage{hyperref}

\title{Symulacja ruchu drogowego na IV obwodnicy Krakowa}
\author{Szymon Gałuszka, Michał Worsowicz, Maciej Nalepa}
\date{\today}
\begin{document}
	\maketitle
	
	\section{Wprowadzenie}
	Symulacja ruchu pozwala znaleźć przyczny utrudnienia ruchu, a tym samym poprawić jakość budowania dróg. Jest ona nierozłącznym narzędziem przy projektowaniu i planowaniu nowych tras i skrzyżowań. Dzięki symulacji można przewidzieć zachowanie pojazdów na przyszłych jezdniach; sprawdzić, czy poradzą sobie z oczekiwanym natężeniem ruchu oraz upewnić się, czy proponowane rozwiązanie nie wywoła kolejnych, nieprzewidzianych utrudnień. \newline
	W ostatnich latach, zarówno w Polsce jak i na świecie, nastąpił rozwój inteligentnych systemów kontroli ruchu ulicznego, które wykorzystują wiele rodzajów danych - zbieranych za pomocą odpowiednich czujników i systemów - aby w  odpowiedni sposób zarządzać sygnalizacją świetlną na skrzyżowaniach i optymalizować przepływ ludzi podróżujących np. samochodem, rowerem lub pieszo. \newline Jednak historia symulacji ruchu ulicznego sięga wiele lat wcześniej. Pierwsze modele natężenia ruchu ulicznego pojawiły się już w latach pięćdziesiątych wraz z rosnącym dostępem do pierwszych komputerów. Rosnąca moc obliczeniowa pozwoliła w następnych latach opracowywać coraz bardziej skomplikowane i zaawansowane modele, które uwzględniały wiele zmiennych oraz różniły się od siebie założeniami i sposobami implementacji. Możemy je podzielić na dwa typy: mikro- i makroskopijne. Pierwszy rodzaj modeli symuluje pojedyńcze jednostki, np. samochody, gdzie każda z nich jest reprezentowana przez swoje parametry, takie jak obecna prędkość lub pozycja. Drugi typ modeli - makroskopijny - uwzględnia natomiast zależności pomiędzy właściwościami natężenia ruchu takimi jak gęstość, przepustowość, średnia prędkość ruchu na drodze. Są tu integrowane mikroskopijne modele, ale w sposób, który przekształca charakterystyki z poziomu pojedyńczych jednostek na porównywalne charakterystyki dla całego systemu. Aby w miarodajny sposób przeprowadzić symulację ruchu drogowego w danym terenie,  będzie należało wybrać odpowiedni sposób na realizację rozwiązania problemu.
	
	TODO: coś od siebie + literatura, może być rozwiązanie historyczne
	
	\section{Definicja problemu}
	TODO: co trzeba zrobić, co symulujemy, co chcemy uzyskać
	
	\section{Propozycja rozwiązania}
	TODO: jak to robimy, literatura
	
	\pagebreak
	\begin{thebibliography}{15}
			\bibitem{id}
		One Road, \textit{Symulacje ruchu drogowego},
		\texttt{\href{http://www.oneroad.pl/symulacje-ruchu-drogowego/}{http://www.oneroad.pl/symulacje-ruchu-drogowego/}}
		
		\bibitem{id}
		AUTOR, \textit{TYTUŁ},
		\texttt{\href{LINK}{nazwa linku}}
		
	\end{thebibliography}
	
\end{document}

% Przegląd literatury i ogólnie pojęty research
\documentclass[a4paper,12pt]{article}
\usepackage[polish]{babel}
\usepackage[utf8]{inputenc}
\usepackage[T1]{fontenc}
\usepackage{indentfirst}
\usepackage[top=2.5cm, bottom=2.5cm, left=2.5cm, right=2.5cm]{geometry}
\usepackage{amsmath}
\usepackage{hyperref}

\title{Symulacja ruchu drogowego na IV obwodnicy Krakowa}
\author{Szymon Gałuszka, Michał Worsowicz, Maciej Nalepa}
\date{\today}
\begin{document}
	\maketitle
	
	\section{Wprowadzenie}
	Symulacja ruchu pozwala znaleźć przyczny utrudnienia ruchu, a tym samym poprawić jakość budowania dróg. [cytat]
	
	TODO: coś od siebie + literatura, może być rozwiązanie historyczne
	
	\section{Definicja problemu}
	TODO: co trzeba zrobić, co symulujemy, co chcemy uzyskać
	
	\section{Propozycja rozwiązania}
	TODO: jak to robimy, literatura
	
	\pagebreak
	\begin{thebibliography}{15}
		
		\bibitem{id}
		AUTOR, \textit{TYTUŁ},
		\texttt{\href{LINK}{nazwa linku}}
		
	\end{thebibliography}
	
\end{document}
